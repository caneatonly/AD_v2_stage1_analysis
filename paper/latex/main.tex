% Ocean Engineering (Elsevier) submission skeleton based on the official
% elsarticle templates placed under `paper/elsarticle/`.
%
% Source of truth: edit `paper/sections/*.md`, then propagate changes here.
%
% Note: this project intentionally keeps the LaTeX tree buildable from the repo root.
\documentclass[preprint,review,12pt]{elsarticle}

%% If you want single spacing (later / final polishing), drop the 'review' option:
%% \documentclass[preprint,12pt]{elsarticle}

%% For final journal layout, elsarticle supports e.g.:
%% \documentclass[final,5p,times]{elsarticle}

%% The amssymb package provides various useful mathematical symbols.
\usepackage{amssymb}
%% The amsmath package provides various useful equation environments.
\usepackage{amsmath}

%% Line numbers (recommended for review).
\usepackage{lineno}

%% For figures, graphicx is loaded by elsarticle.cls; keep it explicit for clarity.
\usepackage{graphicx}
\usepackage{booktabs}
\usepackage{siunitx}
\usepackage{hyperref}
\usepackage{url}

\makeatletter
% Allow building both from repo root and from within `paper/latex/`.
\def\input@path{{paper/latex/}{paper/latex/sections/}}
\makeatother

\modulolinenumbers[5]
\journal{Ocean Engineering}

\begin{document}
\begin{frontmatter}

\title{Design and Characterization of a Mission-Oriented Self-Suspending Underwater Platform with Anisotropic Permeability-Corrected Dynamics and Experimental Identification}

% TODO: Replace with real author list and affiliations (official elsarticle style).
\author[inst1]{First Author\corref{cor1}}
\author[inst1]{Second Author}
\cortext[cor1]{Corresponding author.}
\ead{corresponding.author@example.com}
\affiliation[inst1]{organization={Affiliation},
            addressline={},
            city={},
            postcode={},
            state={},
            country={}}

\begin{abstract}
Underwater corner reflectors are used as passive high-contrast markers for marine detection and localization tasks, and their deployment reliability depends on the carrier platform. This paper presents the design and characterization of a mission-oriented self-suspending underwater platform developed for corner-reflector deployment. The platform is required to pass through a horizontal-launch to vertical-stabilization transition before reaching its working posture; therefore, platform architecture, mass-buoyancy arrangement, and deployment workflow are treated as primary engineering contributions.

For this transition phase, an anisotropic permeability-corrected 3-DOF dynamic model is developed with three physically interpretable coupling parameters, \texttt{mu\_x}, \texttt{mu\_z}, and \texttt{mu\_theta}, to represent directional internal-water coupling effects. CFD-derived static hydrodynamic mappings are used as model evidence for force and moment replacement logic, while rotational damping is retained as an identification-oriented term. A closed validation chain is established through free-decay identification, out-of-sample simulation-to-experiment evaluation, and sensitivity-based envelope analysis. The identification and evaluation stages use consistent initialization and condition-level split rules to reduce leakage risk and support fair sim-real comparison.

The resulting platform-model framework provides consistent reconstruction of transition behavior and practical guidance for deployment-oriented design and parameter selection. Conclusions are explicitly limited to the current geometry, mass distribution, and actuation strategy, and the workflow is organized for reproducibility through protocolized preprocessing and condition-level split control.
\end{abstract}

%% Graphical abstract is optional for this journal and is not included here.

%%Research highlights (also often uploaded as a separate file during submission).
\begin{highlights}
\item Mission-oriented self-suspending platform for corner-reflector deployment
\item 3-DOF anisotropic permeability-corrected dynamics with interpretable coupling terms
\item CFD used as a model-evidence bridge for static coefficient mapping and CP migration
\item Free-decay ID with protocolized splits and out-of-sample sim-real evaluation
\item Sensitivity envelopes for design margins and deployment strategy selection


\end{highlights}

\begin{keyword}
Underwater platform \sep Mission-oriented deployment \sep Free-decay identification \sep CFD evidence \sep Added mass \sep Permeability correction
\end{keyword}

\end{frontmatter}

\linenumbers

% Converted from: paper/sections/01_introduction.md
\section{Introduction}

\subsection{Engineering Background and Motivation}

Underwater corner reflectors serve as passive markers in marine detection and localization tasks. Deployment reliability depends on both the reflector and the carrier platform dynamics during posture transition. In the present deployment scenario, the platform is released near horizontal and is required to stabilize to a near-vertical working posture. The launch-to-stabilization transition is therefore treated as the primary engineering problem.

\subsection{Related Work and Gap Identification}

Existing underwater dynamics studies employ established approaches for hydrodynamic modeling and parameter identification, but most are developed for general-purpose vehicles or for method-focused demonstrations. For mission-specific deployment platforms, this separation introduces uncertainty because structural layout, internal-water coupling, and operating protocol are strongly coupled. Parameter transferability also degrades when large-attitude transition is intrinsic to operation. A suitable framework therefore requires integrated platform design, physically grounded dynamics, and experiment-based validation \citep{R2,R3,R4,R5,R6}.

\subsection{Objectives and Contributions}

This study addresses platform design and dynamic characterization in a single workflow. A mission-oriented self-suspending platform is developed with explicit constraints on structure, mass-buoyancy distribution, and deployment sequence. For the transition phase, an anisotropic permeability-corrected 3-DOF model is formulated. The parameters $\mu_x$, $\mu_z$, and $\mu_\theta$ represent directional internal-water coupling in surge, heave, and pitch.

Figure~\ref{fig:mission-architecture} summarizes the platform architecture and deployment workflow used in this study.

Static CFD is used to obtain hydrodynamic coefficient mappings required for model closure. Rotational damping is retained as an identified term. Model parameters are determined from free-decay data, followed by out-of-sample simulation-to-experiment validation and parametric design analysis.

The contributions are threefold: (1) design, implementation, and experimental characterization of a mission-oriented self-suspending deployment platform; (2) an anisotropic permeability-corrected 3-DOF model with CFD-supported hydrodynamic closure; and (3) an evidence chain from parameter determination to out-of-sample validation and design-oriented parametric analysis.

\subsection{Scope and Paper Organization}

Conclusions are bounded to the current geometry, mass distribution, and actuation strategy. Section~2 describes the platform configuration and deployment scenario. Section~3 presents model development and parameter determination. Section~4 reports validation against free-decay experiments. Section~5 presents parametric analysis and design guidance. Section~6 summarizes conclusions, limitations, and future work.

\begin{figure}[htbp]
	\centering
	% TODO: Insert Fig. 1 here (see paper/figures/FIGURE_REQUIREMENTS.md).
	\caption{Mission-oriented platform architecture and deployment workflow.}
	\label{fig:mission-architecture}
\end{figure}

% Converted from: paper/sections/02_system_and_mission_scenario.md
\section{System and mission scenario}

\subsection{Mission role and engineering motivation}

The platform in this study is developed for underwater corner-reflector deployment. In this context, the corner reflector acts as a passive high-contrast marker to support marine detection and localization-related operations. The engineering challenge is not only payload placement, but also reliable posture transition of the carrier after release. The platform must be deployable in a compact launch state and then autonomously reach a stable in-water working posture with repeatable dynamics.

Unlike a general-purpose underwater vehicle study, this work is mission-oriented from the beginning: the platform is designed around a specific deployment requirement, and the transition dynamics are investigated because they directly affect deployment success and operational readiness. To satisfy confidentiality constraints of mission details, payload-specific tactical information is abstracted, while all dynamic and identification assumptions relevant to reproducibility are retained.

\subsection{Platform architecture and design rationale}

The proposed system is a self-suspending underwater platform with coupled structural and hydrostatic design. The architecture is defined by: (i) rigid-body mass distribution, (ii) buoyancy-center placement relative to the body frame, (iii) internal-water coupling represented later by permeability-corrected terms, and (iv) optional cable-related restoring effects used in the present hardware configuration.

For the nominal setup used throughout this manuscript, key physical parameters are frozen in \path{sim_flip/configs/params_nominal.yaml}: dry mass $m_{dry}=\SI{2.55}{kg}$, wet mass $m_{wet}=\SI{2.76}{kg}$, inner-water equivalent mass $m_{water,inner}=\SI{0.21}{kg}$, pitch inertia $I_{yy}=\SI{0.05741}{kg\,m^2}$, and inner-water inertia $I_{water,inner}=\SI{0.01119}{kg\,m^2}$. Buoyancy is implemented as equivalent mass with $B_{mass}=\SI{2.55}{kg}$ and nominal offset $x_b=\SI{0.02535}{m}$. These values define the baseline configuration for all reported simulation-experiment comparisons.

\begin{figure}[htbp]
  \centering
  % TODO: Insert Fig. 2 here (see paper/figures/FIGURE_REQUIREMENTS.md).
  \caption{Body-axis definition, sign convention, and state interface ($u$, $w$, $q$, $\theta$).}
  \label{fig:conventions}
\end{figure}

\subsection{Mission phase of interest and state boundary}

This paper focuses on one mission-critical phase: horizontal launch to vertical stabilization. The stabilized posture is defined near $\theta=\SI{90}{deg}$ under the project convention. The analysis excludes full closed-loop depth-keeping and long-horizon mission execution; those are treated as future work.

The state interface used across data processing, modeling, and simulation is fixed as $\nu=[u,w,q]^T$ and $\eta=[\theta]$. Sign and axis conventions are frozen in \path{sim_flip/src/conventions.py}: body $x_b$ forward, $z_b$ downward, nose-up pitch positive, and $q=d\theta/dt$ about $+y_b$. Angle of attack is computed as $\alpha=\mathrm{atan2}(w,u)$. Freezing these definitions prevents silent inconsistency between CFD tables, identification scripts, and dynamic simulation.

\subsection{Deployment sequence and experimental program}

The operational sequence used for the current dataset is: initial vertical rest, data recording start, single manual excitation, free decay, and final rest. One raw run contains exactly one excitation-decay event. This rule is enforced in the data protocol to support deterministic segmentation and condition-level split control.

The use of protocolized run definitions, deterministic segmentation, and condition-level split control is consistent with best practices in recent CFD-identification and experimental-validation studies \citep{R3,R4,R5,R6}.

At the time of this draft, the baseline evidence includes three free-decay segments, with planned expansion to a 12-condition matrix over $\theta_0$ and $q_0$ levels (minimum two repeats per condition). Data split is performed by condition blocks rather than random sample points, and leave-one-theta-level-out validation is adopted to test extrapolation behavior. This design links platform operation constraints with identification and out-of-sample evaluation requirements.

\begin{figure}[htbp]
  \centering
  % TODO: Insert Fig. 3 here (see paper/figures/FIGURE_REQUIREMENTS.md).
  \caption{Experimental condition matrix, repeat policy, and anti-leakage split strategy.}
  \label{fig:condition-matrix}
\end{figure}

\subsection{Scope, assumptions, and transferability}

Three assumptions delimit the claims in this paper. First, mission description is abstracted, but the deployment role of the platform is explicit. Second, conclusions are restricted to the present geometry, mass distribution, and actuation strategy. Third, CFD is used as a mechanism and coefficient evidence layer for model-term mapping, not as a standalone novelty claim.

Under these assumptions, the contribution of this section is to define the system boundary and engineering context for the dynamic model and identification framework introduced in the following sections.

% Converted from: paper/sections/03_anisotropic_permeability_corrected_dynamics.md
\section{Dynamic model formulation}

\subsection{Governing equations}

This section introduces the 3-DOF governing model for the horizontal-launch to vertical-stabilization phase, with frozen interfaces $\nu=[u,w,q]^T$ and $\eta=[\theta]$. The body-axis and sign conventions follow Section~2 (Fig.~\ref{fig:conventions}): $x_b$ forward, $z_b$ downward, nose-up pitch positive, and $q=d\theta/dt$ about $+y_b$.

The simulation state is
\begin{equation}
	\mathbf{x} = [u,\; w,\; q,\; \theta]^T,
\end{equation}
where $u$ and $w$ are body-frame translational velocities, $q$ is pitch rate, and $\theta$ is the pitch angle. The kinematic closure is $\dot{\theta}=q$.

For the present phase of interest, the governing dynamics are written in a compact planar Fossen-style form with effective inertias (defined in Section~\ref{sec:added-mass}):
\begin{align}
	m_x\,\dot{u} - m_z\,wq & = X_{cfd}(\alpha) - (W - B)\,\sin\theta + T,                                           \\
	m_z\,\dot{w} + m_x\,uq & = Z_{cfd}(\alpha) + (W - B)\,\cos\theta,                                               \\
	I_y\,\dot{q}           & = M_{cfd}(\alpha) + M_{damp}(q) + M_{bg}(\theta) + M_{cable}(\theta,q) + M_{thruster}, \\
	\dot{\theta}           & = q.
\end{align}

Here, $W=m_{dry}g$ is the weight magnitude and $B=B_{mass}g$ is the buoyancy magnitude (implemented as equivalent mass in the parameter tree). The terms $T$ and $M_{thruster}$ are optional actuation inputs; in the free-decay identification setting they are set to zero.

Hydrodynamic forces and moment are represented by static-CFD tables through a dynamic-pressure scaling:
\begin{equation}
	V = \sqrt{u^2 + w^2}, \qquad Q = \tfrac{1}{2}\rho V^2,
\end{equation}
with a numerical safeguard $Q=0$ when $V<V_{eps}$ for robustness. The CFD-driven force and moment are
\begin{align}
	X_{cfd} & = Q\,A_{ref}\,C_X(\alpha),          \\
	Z_{cfd} & = Q\,A_{ref}\,C_Z(\alpha),          \\
	M_{cfd} & = Q\,A_{ref}\,L_{ref}\,C_m(\alpha),
\end{align}
where $A_{ref}$ and $L_{ref}$ are fixed reference area and length. The attack angle is computed by the frozen convention
\begin{equation}
	\alpha_{raw} = \mathrm{atan2}(w, u),
\end{equation}
and mapped/clamped to the CFD lookup range at runtime (details in Section~4). Following the equation-governance rule, no additional explicit Munk-moment term is introduced when $C_m(\alpha)$ already encodes the corresponding static effect.

\subsection{Anisotropic permeability correction}
\label{sec:added-mass}

The main modeling novelty in this section is an anisotropic permeability correction to the added-mass / added-inertia totals. We conceptually separate (i) an outer-fluid added-mass set and (ii) an inner-water coupling contribution. Three directional coupling parameters
\begin{equation}
	\mu_x,\;\mu_z,\;\mu_{\theta} \in [0,1]
\end{equation}
represent the fraction of inner-water inertia that effectively couples to the rigid-body motion in surge ($x$), heave ($z$), and pitch ($\theta$) channels, respectively. This formulation is intentionally compact and physically interpretable: $\mu \approx 0$ indicates weak coupling (high internal permeability / relative flow), while $\mu \approx 1$ indicates strong coupling (near-lumped inner water).

Using the project sign convention (typical Fossen form expects negative added-mass derivatives), the total added-mass / added-inertia terms are defined as:
\begin{align}
	X_{\dot{u},total} & = X_{\dot{u},outer} - \mu_x\,m_{water,inner},        \\
	Z_{\dot{w},total} & = Z_{\dot{w},outer} - \mu_z\,m_{water,inner},        \\
	M_{\dot{q},total} & = M_{\dot{q},outer} - \mu_{\theta}\,I_{water,inner}.
\end{align}

The corresponding effective inertias used in the ODE denominators are
\begin{equation}
	m_x = m_{dry} - X_{\dot{u},total}, \qquad
	m_z = m_{dry} - Z_{\dot{w},total}, \qquad
	I_y = I_{yy} - M_{\dot{q},total}.
\end{equation}

All terms above are frozen by the parameter contract and grouped into rigid-body, outer-added-mass, and permeability categories in the project configuration. Consistent with the implementation, the translational denominators use $m_{dry}$ (not $m_{wet}$).

\subsection{Restoring and damping terms}

The model retains Fossen-style rigid-body coupling through the Coriolis-like cross terms $\pm m\,\cdot\,(\cdot)\,q$ in the surge/heave equations. Restoring, damping, and optional configuration-dependent terms are defined as follows.

Hydrostatic restoring moment is modeled by a buoyancy-center offset $(x_b,z_b)$ in the body frame:
\begin{equation}
	M_{bg}(\theta) = B\,(z_b\,\sin\theta + x_b\,\cos\theta),
\end{equation}
where $B=B_{mass}g$.

Rotational damping is retained as an identification-oriented term (not obtained from static CFD tables):
\begin{equation}
	M_{damp}(q) = -\left(d_q + d_{q,abs}|q|\right)q.
\end{equation}

If a cable restoring mechanism is enabled in the hardware configuration, its contribution is modeled as a linear torsional spring-damper around an equilibrium angle $\theta_{eq}$:
\begin{equation}
	M_{cable}(\theta,q) = -K_{cable}(\theta - \theta_{eq}) - C_{cable,q}\,q.
\end{equation}

\subsection{Implementation summary}

The manuscript equations above map one-to-one to the repository implementation:
\begin{itemize}
	\item State ordering: $y=[u,w,q,\theta]$ is used consistently in the ODE right-hand side and event functions.
	\item Kinematics and CFD inputs: $V$, $Q$, and $\alpha$ are computed from the body-frame velocities; coefficient interpolation uses the angle-dependent lookup table.
	\item Permeability correction: added-mass totals and effective inertias are derived from the parameter tree.
	\item All physical parameters ($\rho$, $g$, $A_{ref}$, $L_{ref}$, $m_{dry}$, $I_{yy}$, $m_{water,inner}$, $I_{water,inner}$, $X_{\dot{u},outer}$, $Z_{\dot{w},outer}$, $M_{\dot{q},outer}$, $\mu_x$, $\mu_z$, $\mu_\theta$, $B_{mass}$, $x_b$, $z_b$, $d_q$, $d_{q,abs}$, and optional cable terms) are single-sourced in the project configuration file.
\end{itemize}

\begin{figure}[htbp]
	\centering
	% TODO: Insert Fig. 4 here (see paper/figures/FIGURE_REQUIREMENTS.md).
	\caption{Model-term map from physical mechanisms to equation blocks and code interfaces.}
	\label{fig:model-term-map}
\end{figure}



% TODO: Convert the remaining section drafts when ready.
% Placeholder conversion from: paper/sections/04_cfd_to_model_mapping_and_validation.md
\section{CFD-based hydrodynamic characterization}

\subsection{Simulation setup and methodology}
CFD is used as a mechanism and coefficient evidence layer for static hydrodynamic mappings and model-term replacement logic.

\subsection{Mesh independence and convergence}
Mesh independence and residual-convergence summaries are reported to establish numerical reliability for the coefficient tables used by the dynamics module.

\begin{figure}[htbp]
	\centering
	% TODO: Insert Fig. 5 here.
	\caption{CFD credibility summary: mesh refinement and residual convergence.}
	\label{fig:cfd-credibility}
\end{figure}

\subsection{Hydrodynamic coefficients}
Full-angle mappings of $C_X(\alpha)$, $C_Z(\alpha)$, and $C_m(\alpha)$ are presented, including zoom-in panels for high-angle regions where transition dynamics are most sensitive.

\begin{figure}[htbp]
	\centering
	% TODO: Insert Fig. 6 here.
	\caption{Full-AoA $C_X/C_Z/C_m$ curves with high-AoA local enlargement.}
	\label{fig:cfd-coeffs}
\end{figure}

\subsection{Pressure-center migration}
$X_{cp}(\alpha)$ is analyzed with static stable/unstable region annotations to support moment interpretation during large-attitude transition.

\begin{figure}[htbp]
	\centering
	% TODO: Insert Fig. 7 here.
	\caption{Pressure-center migration $X_{cp}(\alpha)$ with static stability-region annotations.}
	\label{fig:xcp}
\end{figure}

\subsection{Flow visualization}
Selected flow and pressure snapshots at key angles are provided as qualitative mechanism evidence.

\begin{figure}[htbp]
	\centering
	% TODO: Insert Fig. 8 here.
	\caption{Representative flow and pressure snapshots at $\alpha=\SI{0}{deg}/\SI{30}{deg}/\SI{60}{deg}/\SI{90}{deg}$.}
	\label{fig:flow-snapshots}
\end{figure}


% Placeholder conversion from: paper/sections/05_parameter_identification_free_decay.md
\section{Parameter identification from free-decay tests}

\subsection{Protocol and data governance}
This section documents preprocessing, segmentation, condition-level split constraints, and anti-leakage rules under a protocolized identification configuration and experiment manifest.

\subsection{Joint identification workflow}
The workflow includes robust fitting, multi-start optimization, and optional bootstrap confidence analysis over selected segments.

\subsection{Diagnostic outputs and confidence}
Residual checks, convergence behavior, and bootstrap confidence intervals are reported to evaluate parameter identifiability and stability.

\begin{figure}[htbp]
	\centering
	% TODO: Insert Fig. 9 here.
	\caption{Identification diagnostics: residuals, multi-start convergence, and bootstrap confidence intervals.}
	\label{fig:id-diagnostics}
\end{figure}


% Placeholder conversion from: paper/sections/06_sim_real_validation.md
\section{Simulation-to-experiment validation}

\subsection{Evaluation setup}
This section reports in-sample and out-of-sample simulation-to-experiment validation using the same initial-condition and parameter loading rules defined in the identification protocol. For each run, $\theta_0$ is initialized from the first sample point of the corresponding segment, and simulated trajectories are interpolated onto experimental timestamps before metric calculation.

\subsection{Time-series comparison}
Representative runs are reported in fixed order: in-sample group first, out-of-sample group second. Both $\theta(t)$ and $q(t)$ are shown whenever angular-rate sensing is available, with synchronized time axes and consistent sign conventions.

\begin{figure}[htbp]
  \centering
  % TODO: Insert Fig. 10 here.
  \caption{Multi-run in-sample and out-of-sample sim-real trajectories ($\theta$, $q$).}
  \label{fig:sim-real-traj}
\end{figure}

\begin{figure}[htbp]
  \centering
  % TODO: Insert Fig. 11 here.
  \caption{Time-domain error traces with zero-reference and RMSE bands.}
  \label{fig:error-traces}
\end{figure}

\subsection{Metrics and interpretation}
The mandatory absolute metrics are reported for each split: $TAAE_\theta$, $TASE_\theta$, $RMSE_\theta$, $MAE_\theta$, $MaxAbs_\theta$, $dt_{90}$, overshoot error, steady-window mean and standard deviation, $RMSE_q$, and $MAE_q$. GapScore-based ranking is reported as a secondary aggregate indicator and interpreted jointly with absolute metrics.

\begin{table}[htbp]
  \centering
  \caption{Validation metrics by condition and split.}
  \label{tab:validation-metrics}
  \begin{tabular}{ll}
    \toprule
    Metric & Value \\
    \midrule
    TODO & TODO \\
    \bottomrule
  \end{tabular}
\end{table}


% Placeholder conversion from: paper/sections/07_sensitivity_and_design_implications.md
\section{Sensitivity and design implications}

\subsection{Initial-energy envelope and phase portraits}
The initial-condition envelope is scanned to identify success/failure regions for transition completion, stall risk, and roll-over tendency.

\subsection{Thrust timing and energy cutoff strategy}
Shutdown-angle sweeps are compared against always-on strategy, with response time, overshoot, terminal rate, and safety displacement indicators.

\subsection{BG and damping robustness}
Sensitivity to $BG$ variation and damping uncertainty is quantified to derive practical design margins and degradation trends.

\begin{figure}[htbp]
  \centering
  % TODO: Insert Fig. 12 here.
  \caption{Condition-normalized metric heatmap and GapScore ranking.}
  \label{fig:gapscore}
\end{figure}

\begin{figure}[htbp]
  \centering
  % TODO: Insert Fig. 13 here.
  \caption{Sensitivity envelopes for initial energy, $BG$, and damping uncertainty.}
  \label{fig:envelopes}
\end{figure}

\begin{figure}[htbp]
  \centering
  % TODO: Insert Fig. 14 here.
  \caption{Strategy surface $J(u_0,\theta_{off})$ with feasible valley regions.}
  \label{fig:strategy-surface}
\end{figure}


\input{sections/08_conclusion_and_outlook}

\section*{Data availability}
% TODO: Fill per journal policy (or remove if not applicable).

\section*{CRediT authorship contribution statement}
% TODO: Fill per Ocean Engineering Guide for Authors (CRediT roles).

\section*{Declaration of competing interest}
The authors declare that they have no known competing financial interests or personal relationships that could have appeared to influence the work reported in this paper.

\section*{Acknowledgements}
% TODO: Place acknowledgements in this section, directly before the reference list.

% NOTE: Enforce local official Ocean Engineering-compatible numeric style.
% Strictly use the BST file under `paper/elsarticle/`.
% NOTE: Use workspace-relative paths so latexmk can pre-check file existence.
\bibliographystyle{paper/elsarticle/elsarticle-num}
\bibliography{paper/latex/references}

\end{document}
