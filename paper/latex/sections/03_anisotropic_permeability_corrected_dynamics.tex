% Converted from: paper/sections/03_anisotropic_permeability_corrected_dynamics.md
\section{Dynamic Model Development and Parameter Determination}

\subsection{Governing Equations}

This section presents a 3-DOF governing model for the horizontal-launch to vertical-stabilization phase, with state interfaces $\nu=[u,w,q]^T$ and $\eta=[\theta]$. The body-axis convention follows Section~2 and Fig.~\ref{fig:conventions}: $x_b$ forward, $z_b$ downward, positive pitch nose-up, and $q=d\theta/dt$ about $+y_b$.

The simulation state is
\begin{equation}
	\mathbf{x} = [u,\; w,\; q,\; \theta]^T,
\end{equation}
where $u$ and $w$ are body-frame translational velocities, $q$ is pitch rate, and $\theta$ is pitch angle. The kinematic closure is $\dot{\theta}=q$.

For the present phase of interest, the governing dynamics are written in a compact planar Fossen-style form with effective inertias (defined in Section~3.2):
\begin{align}
	m_x\,\dot{u} - m_z\,wq & = X_{cfd}(\alpha) - (W - B)\,\sin\theta + T,                                           \\
	m_z\,\dot{w} + m_x\,uq & = Z_{cfd}(\alpha) + (W - B)\,\cos\theta,                                               \\
	I_y\,\dot{q}           & = M_{cfd}(\alpha) + M_{damp}(q) + M_{bg}(\theta) + M_{cable}(\theta,q) + M_{thruster}, \\
	\dot{\theta}           & = q.
\end{align}

Here, $W=m_{dry}g$ is the weight magnitude and $B=B_{mass}g$ is the buoyancy magnitude. The terms $T$ and $M_{thruster}$ are optional actuation inputs; they are set to zero in free-decay parameter determination.

Hydrodynamic forces and moment are represented by static mappings through dynamic-pressure scaling:
\begin{equation}
	V = \sqrt{u^2 + w^2}, \qquad Q = \tfrac{1}{2}\rho V^2,
\end{equation}
with a numerical safeguard $Q=0$ when $V<V_{eps}$. The model uses
\begin{align}
	X_{cfd} & = Q\,A_{ref}\,C_X(\alpha),          \\
	Z_{cfd} & = Q\,A_{ref}\,C_Z(\alpha),          \\
	M_{cfd} & = Q\,A_{ref}\,L_{ref}\,C_m(\alpha).
\end{align}

The attack angle is computed by
\begin{equation}
	\alpha_{raw} = \mathrm{atan2}(w, u),
\end{equation}
and mapped/clamped to the lookup range at runtime.

\subsection{Anisotropic Permeability Correction}

The model introduces an anisotropic permeability correction to the added-mass and added-inertia terms. Two contributions are considered: (i) outer-fluid added mass and (ii) inner-water coupling. Three directional coupling parameters
\begin{equation}
	\mu_x,\;\mu_z,\;\mu_{\theta} \in [0,1]
\end{equation}
represent the fraction of inner-water inertia coupled to rigid-body motion in surge, heave, and pitch, respectively.

Using the project sign convention (Fossen-style negative added-mass derivatives), total added terms are
\begin{align}
	X_{\dot{u},total} & = X_{\dot{u},outer} - \mu_x\,m_{water,inner},        \\
	Z_{\dot{w},total} & = Z_{\dot{w},outer} - \mu_z\,m_{water,inner},        \\
	M_{\dot{q},total} & = M_{\dot{q},outer} - \mu_{\theta}\,I_{water,inner}.
\end{align}

The corresponding effective inertias in the ODE denominators are
\begin{equation}
	m_x = m_{dry} - X_{\dot{u},total}, \qquad
	m_z = m_{dry} - Z_{\dot{w},total}, \qquad
	I_y = I_{yy} - M_{\dot{q},total}.
\end{equation}

All terms above are computed in the project parameter pipeline. Consistent with implementation, translational denominators use $m_{dry}$ (not $m_{wet}$).

\begin{figure}[htbp]
	\centering
	% TODO: Insert Fig. 4 here (see paper/figures/FIGURE_REQUIREMENTS.md).
	\caption{Model-term map from physical mechanisms to equation blocks and code interfaces.}
	\label{fig:sec3-model-term-map}
\end{figure}

The force and moment decomposition in the body-fixed frame is summarized in Fig.~\ref{fig:sec3-fbd}.

\begin{figure}[htbp]
	\centering
	% TODO: Insert Fig. 5 here.
	\caption{Free-body force and moment decomposition for the transition-phase model.}
	\label{fig:sec3-fbd}
\end{figure}

\begin{figure}[htbp]
	\centering
	% TODO: Insert Fig. 6 here.
	\caption{Conceptual interpretation of anisotropic internal-water coupling parameters ($\mu_x$, $\mu_z$, $\mu_\theta$).}
	\label{fig:sec3-aniso-coupling-concept}
\end{figure}

\subsection{Hydrodynamic Coefficient Acquisition from CFD}

CFD is used to close the dynamic model by supplying static mappings $C_X(\alpha)$, $C_Z(\alpha)$, $C_m(\alpha)$, and the pressure-center trend $X_{cp}(\alpha)$ for mechanism interpretation. CFD is positioned as supporting evidence for model closure.

Numerical verification is provided through mesh-independence and residual-convergence summaries to support coefficient reliability.

\begin{figure}[htbp]
	\centering
	% TODO: Insert Fig. 7 here.
	\caption{CFD verification results: mesh-independence and residual convergence.}
	\label{fig:sec3-cfd-credibility}
\end{figure}

The coefficient mappings are reported across the full attack-angle range required for transition simulation, with local enlargement in high-angle regions where response sensitivity is strongest.

\begin{figure}[htbp]
	\centering
	% TODO: Insert Fig. 8 here.
	\caption{Hydrodynamic coefficient mappings versus attack angle ($C_X$, $C_Z$, $C_m$).}
	\label{fig:sec3-cfd-coeffs}
\end{figure}

Pressure-center migration is then analyzed to interpret pitch-moment behavior and static stability-region transition.

\begin{figure}[htbp]
	\centering
	% TODO: Insert Fig. 9 here.
	\caption{Pressure-center migration and static-stability-region transition versus attack angle.}
	\label{fig:sec3-xcp}
\end{figure}

Representative flow and pressure fields at selected angles are finally presented to support the observed coefficient trends and pressure-center movement.

\begin{figure}[htbp]
	\centering
	% TODO: Insert Fig. 10 here.
	\caption{Representative pressure and velocity field patterns at selected attack angles.}
	\label{fig:sec3-flow-snapshots}
\end{figure}

\subsection{Parameter Identification Procedure}

After model structure and coefficient mappings are fixed, parameters are determined using a protocolized free-decay identification procedure. The workflow includes preprocessing, deterministic segmentation, condition-level split control, energy-based damping estimation, and ODE residual minimization with multi-start initialization. Bootstrap analysis is included when uncertainty quantification is required.

\begin{figure}[htbp]
	\centering
	% TODO: Insert Fig. 11 here.
	\caption{Parameter-identification workflow from segmented free-decay data to calibrated model parameters.}
	\label{fig:sec3-id-workflow}
\end{figure}

The identification stage reports residual diagnostics, convergence behavior across starts, and confidence summaries to assess parameter stability and identifiability.

\begin{figure}[htbp]
	\centering
	% TODO: Insert Fig. 12 here.
	\caption{Identification diagnostics and uncertainty summaries (residuals, multi-start convergence, bootstrap intervals).}
	\label{fig:sec3-id-diagnostics}
\end{figure}

\subsection{Identified Parameter Set and Model Closure}

Section~3 yields a closed model package containing (i) fixed structural and hydrostatic parameters, (ii) CFD-supported static hydrodynamic mappings, and (iii) identified dynamic parameters for transition-phase simulation. This package is used without re-tuning in Section~4 for validation against free-decay experiments.
