% Converted from: paper/sections/03_anisotropic_permeability_corrected_dynamics.md
\section{Dynamic model formulation}

\subsection{Governing equations}

This section introduces the 3-DOF governing model for the horizontal-launch to vertical-stabilization phase, with frozen interfaces $\nu=[u,w,q]^T$ and $\eta=[\theta]$. The body-axis and sign conventions follow Section~2 (Fig.~\ref{fig:conventions}): $x_b$ forward, $z_b$ downward, nose-up pitch positive, and $q=d\theta/dt$ about $+y_b$.

The simulation state is
\begin{equation}
	\mathbf{x} = [u,\; w,\; q,\; \theta]^T,
\end{equation}
where $u$ and $w$ are body-frame translational velocities, $q$ is pitch rate, and $\theta$ is the pitch angle. The kinematic closure is $\dot{\theta}=q$.

For the present phase of interest, the governing dynamics are written in a compact planar Fossen-style form with effective inertias (defined in Section~\ref{sec:added-mass}):
\begin{align}
	m_x\,\dot{u} - m_z\,wq & = X_{cfd}(\alpha) - (W - B)\,\sin\theta + T,                                           \\
	m_z\,\dot{w} + m_x\,uq & = Z_{cfd}(\alpha) + (W - B)\,\cos\theta,                                               \\
	I_y\,\dot{q}           & = M_{cfd}(\alpha) + M_{damp}(q) + M_{bg}(\theta) + M_{cable}(\theta,q) + M_{thruster}, \\
	\dot{\theta}           & = q.
\end{align}

Here, $W=m_{dry}g$ is the weight magnitude and $B=B_{mass}g$ is the buoyancy magnitude (implemented as equivalent mass in the parameter tree). The terms $T$ and $M_{thruster}$ are optional actuation inputs; in the free-decay identification setting they are set to zero.

Hydrodynamic forces and moment are represented by static-CFD tables through a dynamic-pressure scaling:
\begin{equation}
	V = \sqrt{u^2 + w^2}, \qquad Q = \tfrac{1}{2}\rho V^2,
\end{equation}
with a numerical safeguard $Q=0$ when $V<V_{eps}$ for robustness. The CFD-driven force and moment are
\begin{align}
	X_{cfd} & = Q\,A_{ref}\,C_X(\alpha),          \\
	Z_{cfd} & = Q\,A_{ref}\,C_Z(\alpha),          \\
	M_{cfd} & = Q\,A_{ref}\,L_{ref}\,C_m(\alpha),
\end{align}
where $A_{ref}$ and $L_{ref}$ are fixed reference area and length. The attack angle is computed by the frozen convention
\begin{equation}
	\alpha_{raw} = \mathrm{atan2}(w, u),
\end{equation}
and mapped/clamped to the CFD lookup range at runtime (details in Section~4). Following the equation-governance rule, no additional explicit Munk-moment term is introduced when $C_m(\alpha)$ already encodes the corresponding static effect.

\subsection{Anisotropic permeability correction}
\label{sec:added-mass}

The main modeling novelty in this section is an anisotropic permeability correction to the added-mass / added-inertia totals. We conceptually separate (i) an outer-fluid added-mass set and (ii) an inner-water coupling contribution. Three directional coupling parameters
\begin{equation}
	\mu_x,\;\mu_z,\;\mu_{\theta} \in [0,1]
\end{equation}
represent the fraction of inner-water inertia that effectively couples to the rigid-body motion in surge ($x$), heave ($z$), and pitch ($\theta$) channels, respectively. This formulation is intentionally compact and physically interpretable: $\mu \approx 0$ indicates weak coupling (high internal permeability / relative flow), while $\mu \approx 1$ indicates strong coupling (near-lumped inner water).

Using the project sign convention (typical Fossen form expects negative added-mass derivatives), the total added-mass / added-inertia terms are defined as:
\begin{align}
	X_{\dot{u},total} & = X_{\dot{u},outer} - \mu_x\,m_{water,inner},        \\
	Z_{\dot{w},total} & = Z_{\dot{w},outer} - \mu_z\,m_{water,inner},        \\
	M_{\dot{q},total} & = M_{\dot{q},outer} - \mu_{\theta}\,I_{water,inner}.
\end{align}

The corresponding effective inertias used in the ODE denominators are
\begin{equation}
	m_x = m_{dry} - X_{\dot{u},total}, \qquad
	m_z = m_{dry} - Z_{\dot{w},total}, \qquad
	I_y = I_{yy} - M_{\dot{q},total}.
\end{equation}

All terms above are frozen by the parameter contract and grouped into rigid-body, outer-added-mass, and permeability categories in the project configuration. Consistent with the implementation, the translational denominators use $m_{dry}$ (not $m_{wet}$).

\subsection{Restoring and damping terms}

The model retains Fossen-style rigid-body coupling through the Coriolis-like cross terms $\pm m\,\cdot\,(\cdot)\,q$ in the surge/heave equations. Restoring, damping, and optional configuration-dependent terms are defined as follows.

Hydrostatic restoring moment is modeled by a buoyancy-center offset $(x_b,z_b)$ in the body frame:
\begin{equation}
	M_{bg}(\theta) = B\,(z_b\,\sin\theta + x_b\,\cos\theta),
\end{equation}
where $B=B_{mass}g$.

Rotational damping is retained as an identification-oriented term (not obtained from static CFD tables):
\begin{equation}
	M_{damp}(q) = -\left(d_q + d_{q,abs}|q|\right)q.
\end{equation}

If a cable restoring mechanism is enabled in the hardware configuration, its contribution is modeled as a linear torsional spring-damper around an equilibrium angle $\theta_{eq}$:
\begin{equation}
	M_{cable}(\theta,q) = -K_{cable}(\theta - \theta_{eq}) - C_{cable,q}\,q.
\end{equation}

\subsection{Implementation summary}

The manuscript equations above map one-to-one to the repository implementation:
\begin{itemize}
	\item State ordering: $y=[u,w,q,\theta]$ is used consistently in the ODE right-hand side and event functions.
	\item Kinematics and CFD inputs: $V$, $Q$, and $\alpha$ are computed from the body-frame velocities; coefficient interpolation uses the angle-dependent lookup table.
	\item Permeability correction: added-mass totals and effective inertias are derived from the parameter tree.
	\item All physical parameters ($\rho$, $g$, $A_{ref}$, $L_{ref}$, $m_{dry}$, $I_{yy}$, $m_{water,inner}$, $I_{water,inner}$, $X_{\dot{u},outer}$, $Z_{\dot{w},outer}$, $M_{\dot{q},outer}$, $\mu_x$, $\mu_z$, $\mu_\theta$, $B_{mass}$, $x_b$, $z_b$, $d_q$, $d_{q,abs}$, and optional cable terms) are single-sourced in the project configuration file.
\end{itemize}

\begin{figure}[htbp]
	\centering
	% TODO: Insert Fig. 4 here (see paper/figures/FIGURE_REQUIREMENTS.md).
	\caption{Model-term map from physical mechanisms to equation blocks and code interfaces.}
	\label{fig:model-term-map}
\end{figure}

