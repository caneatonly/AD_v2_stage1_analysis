% Placeholder conversion from: paper/sections/06_sim_real_validation.md
\section{Simulation-to-experiment validation}

\subsection{Evaluation setup}
This section reports in-sample and out-of-sample simulation-to-experiment validation using the same initial-condition and parameter loading rules defined in the identification protocol. For each run, $\theta_0$ is initialized from the first sample point of the corresponding segment, and simulated trajectories are interpolated onto experimental timestamps before metric calculation.

\subsection{Time-series comparison}
Representative runs are reported in fixed order: in-sample group first, out-of-sample group second. Both $\theta(t)$ and $q(t)$ are shown whenever angular-rate sensing is available, with synchronized time axes and consistent sign conventions.

\begin{figure}[htbp]
  \centering
  % TODO: Insert Fig. 10 here.
  \caption{Multi-run in-sample and out-of-sample sim-real trajectories ($\theta$, $q$).}
  \label{fig:sim-real-traj}
\end{figure}

\begin{figure}[htbp]
  \centering
  % TODO: Insert Fig. 11 here.
  \caption{Time-domain error traces with zero-reference and RMSE bands.}
  \label{fig:error-traces}
\end{figure}

\subsection{Metrics and interpretation}
The mandatory absolute metrics are reported for each split: $TAAE_\theta$, $TASE_\theta$, $RMSE_\theta$, $MAE_\theta$, $MaxAbs_\theta$, $dt_{90}$, overshoot error, steady-window mean and standard deviation, $RMSE_q$, and $MAE_q$. GapScore-based ranking is reported as a secondary aggregate indicator and interpreted jointly with absolute metrics.

\begin{table}[htbp]
  \centering
  \caption{Validation metrics by condition and split.}
  \label{tab:validation-metrics}
  \begin{tabular}{ll}
    \toprule
    Metric & Value \\
    \midrule
    TODO & TODO \\
    \bottomrule
  \end{tabular}
\end{table}

