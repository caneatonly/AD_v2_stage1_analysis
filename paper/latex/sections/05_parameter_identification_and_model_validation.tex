% Placeholder conversion from: paper/sections/05_parameter_identification_and_model_validation.md
\section{Experimental identification and model validation}

\subsection{Test procedure and data processing}
This section documents preprocessing, segmentation, condition-level split constraints, and anti-leakage rules.

\subsection{Identification method and diagnostics}
The workflow includes robust fitting, multi-start optimization, and optional bootstrap confidence analysis over selected segments. Residual checks, convergence behavior, and bootstrap confidence intervals are reported to evaluate parameter identifiability and stability.

\begin{figure}[htbp]
	\centering
	% TODO: Insert Fig. 9 here.
	\caption{Identification diagnostics: residuals, multi-start convergence, and bootstrap confidence intervals.}
	\label{fig:id-diagnostics}
\end{figure}

\subsection{Simulation--experiment comparison}
In-sample and out-of-sample simulation-to-experiment validation using the same initial-condition and parameter loading rules defined in the identification protocol. For each run, $\theta_0$ is initialized from the first sample point of the corresponding segment, and simulated trajectories are interpolated onto experimental timestamps before metric calculation.

\begin{figure}[htbp]
	\centering
	% TODO: Insert Fig. 10 here.
	\caption{Multi-run in-sample and out-of-sample sim-real trajectories ($\theta$, $q$).}
	\label{fig:sim-real-trajectories}
\end{figure}

\begin{figure}[htbp]
	\centering
	% TODO: Insert Fig. 11 here.
	\caption{Time-domain error traces with zero-reference and RMSE bands.}
	\label{fig:error-traces}
\end{figure}

\subsection{Error analysis}
The mandatory absolute metrics are reported for each split. GapScore-based ranking is reported as a secondary aggregate indicator and interpreted jointly with absolute metrics.

\begin{table}[htbp]
	\centering
	\caption{Validation metrics by condition and split.}
	\label{tab:validation-metrics}
	% TODO: Insert Table 1 here.
\end{table}

\begin{figure}[htbp]
	\centering
	% TODO: Insert Fig. 12 here.
	\caption{Condition-normalized metric heatmap and GapScore ranking.}
	\label{fig:gapscore-heatmap}
\end{figure}
